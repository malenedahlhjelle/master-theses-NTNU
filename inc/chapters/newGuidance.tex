\section{Guidance System}

This chapter describes the implementation of the guidance system, which takes the predefined
path generated in the path frame and rotates the waypoints into the local net plane before
computing the desired velocity commands. The algorithm preserves the structure of the original
Pure Pursuit method, but instead of navigating in the $yz$-plane of NED, it navigates in the 
tangent plane of the net surface. The surge distance control regulating $x_d$ is unchanged.

The vehicle is modeled as a point mass, justified by the fact that the vehicle has full 6DOF
authority. The goal is to follow the rotated path smoothly while keeping the camera facing
the net surface.

\subsection{Coordinate Frame}

We use the standard NED frame:

\[
x \text{ (North)}, \qquad y \text{ (East)}, \qquad z \text{ (Down)}.
\]

The planned inspection path is defined in a \emph{path frame} $\{P\}$, in which all waypoints
$W_i^P$ are generated.

At runtime, the perception system provides an estimate of the local net surface normal:

\[
n_f \in \mathbb{R}^3.
\]

From this, we construct the tangent plane of the net:

\[
t_f = \frac{n_f \times z_n}{\|n_f \times z_n\|}, \qquad 
b_f = n_f \times t_f, \qquad 
z_n = [0, 0, 1]^T.
\]

The pair $(t_f, b_f)$ spans the plane along the net surface.  
The transformation from the path frame to the net plane is represented by

\[
R_{fP} = 
\begin{bmatrix}
t_f & b_f
\end{bmatrix},
\]

which maps $(y_P, z_P)$ coordinates of the path frame into the tangent plane.

\subsection{Algorithm Overview}

Each control cycle of the \texttt{guidance\_node} executes the following steps, identical in 
structure to the original implementation:

\begin{enumerate}
    \item Rotate the next $k$ waypoints from the path frame into the net plane.
    \item Compute the distance from the current position to each rotated waypoint.
    \item Select the closest waypoint and choose a point $n$ steps ahead as the target.
    \item Compute the desired velocity components along the tangent plane.
    \item Compute desired yaw and pitch such that the surge axis aligns with the net normal.
    \item Use the unchanged surge controller to maintain distance $x_d$.
\end{enumerate}

\subsection{Mathematical Formulation}

\subsubsection*{Rotation of Waypoints Into the Net Plane}

Each waypoint is given in the path frame:
\[
W_i^P = [0, \, y_i^P, \, z_i^P]^T.
\]

The corresponding coordinates in the net plane are

\[
W_{i,f} = 
\begin{bmatrix}
t_f & b_f
\end{bmatrix}
\begin{bmatrix}
y_i^P \\ z_i^P
\end{bmatrix}.
\]

This replaces the original projection into the $(y,z)$ plane.

\subsubsection*{Nearest-Point Selection (Pure Pursuit)}

Let the current position expressed in the net plane be

\[
p_f =
\begin{bmatrix}
(p - p_0) \cdot t_f \\[3pt]
(p - p_0) \cdot b_f
\end{bmatrix}.
\]

For each of the next $k$ rotated waypoints $W_{i,f}$ we compute

\[
d_i = \| p_f - W_{i,f} \|.
\]

The nearest waypoint is

\[
j^\star = \arg\min_i d_i.
\]

The Pure Pursuit target point becomes

\[
p_{d,f} = W_{j^\star + n, f}.
\]

\subsubsection*{Tangential Plane Error and Desired Velocity}

The tracking error in the net plane is

\[
e_f = p_f - p_{d,f} =
\begin{bmatrix}
e_t \\ e_b
\end{bmatrix}.
\]

The desired velocity in the tangent plane is

\[
v_{f,d} = -k_g \frac{e_f}{\|e_f\|}.
\]

Transforming this back into NED gives

\[
v_d^n = v_{t,d}\, t_f \;+\; v_{b,d}\, b_f.
\]

These become desired sway and heave after rotation to body frame.

\subsubsection*{Orientation Control Using Net Normal}

Yaw and pitch follow directly from the net normal:

\[
\psi_d = \atan2(n_{f,y},\, n_{f,x}),
\qquad
\theta_d = -\arcsin(n_{f,z}).
\]

\subsubsection*{Surge Control (Unchanged)}

Surge distance control remains identical to the original chapter:

\[
e_x = x - x_d,
\qquad
u_d = -K_{p1} e_x - K_{d1} u.
\]

The full desired NED velocity is

\[
v_d^n =
\begin{bmatrix}
u_d \\
(v_{t,d} t_f + v_{b,d} b_f)_y \\
(v_{t,d} t_f + v_{b,d} b_f)_z
\end{bmatrix}.
\]

\subsection{Summary}

This chapter introduced a modification of the guidance system in which the predefined path in
the path frame is rotated into the local net plane before computing tracking commands. The 
Pure Pursuit structure is preserved, but instead of navigating in the global $yz$-plane, the ROV 
navigates inside the tangent plane of the net surface. The surge distance controller is unchanged,
ensuring correct standoff, while the net normal determines yaw and pitch.
