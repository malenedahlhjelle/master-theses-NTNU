% \section{Net-Relative Motion and Circular Nets}

% In this work, all guidance and control are formulated \textbf{relative to the net}, rather than the 
% global NED frame. The goal is to ensure that the inspection pattern (up–across–down–across) follows 
% the geometry of the net surface even if the cage moves, rotates, or deforms under environmental load.

% \subsection{Net Frame Definition}

% We define a net-fixed coordinate frame $\{\mathcal{N}_\text{net}\}$ that moves with the net:
% \begin{itemize}[noitemsep]
%     \item $x_\text{net}$: outward normal direction (used for standoff control),
%     \item $y_\text{net}$: horizontal direction along the net surface,
%     \item $z_\text{net}$: vertical direction along the net (positive downward).
% \end{itemize}

% In the simplified planar simulation used in this project, the net frame is aligned with NED:
% \[
% x_\text{net} \parallel x_\text{NED},\qquad
% y_\text{net} \parallel y_\text{NED},\qquad
% z_\text{net} \parallel z_\text{NED},
% \]
% but all motion and control laws are interpreted as \emph{net-relative}: the vehicle controls its 
% standoff in $x_\text{net}$ and performs the scanning motion in the $(y_\text{net},z_\text{net})$–plane.

% \subsection{Extension to Circular Nets}

% For a circular cage, the net is modeled as a vertical cylinder. The net-relative position is 
% parameterized by
% \[
% (r,\,\varphi,\,z_\text{net}),
% \]
% where
% \begin{itemize}[noitemsep]
%     \item $r$ is the radial distance to the net (used for standoff),
%     \item $\varphi$ is the angular coordinate around the cage (used for horizontal progression),
%     \item $z_\text{net}$ is the vertical coordinate (used for up/down scanning).
% \end{itemize}

% The scanning pattern \emph{up–across–down–across} is preserved by interpreting:
% \begin{itemize}[noitemsep]
%     \item ``up/down''  → motion in $z_\text{net}$,  
%     \item ``across''   → motion in $\varphi$ (replacing linear $y$),  
%     \item standoff     → radial control $r \rightarrow r_d$.
% \end{itemize}

% The horizontal step size $SW$ in the planar case becomes an angular increment $\Delta\varphi$ for 
% circular nets, so the vehicle moves from $\varphi_d$ to $\varphi_d+\Delta\varphi$ during each 
% horizontal segment.

% \subsection{Net-Relative Guidance and Control}

% Both the Pure Pursuit guidance system and the mode controller operate on states expressed in 
% $\{\mathcal{N}_\text{net}\}$. Pure Pursuit provides desired motion direction in the net plane, while 
% the mode controller enforces structured scanning motion with constant speed along each segment.

% For circular nets the desired net-plane velocity is
% \[
% \mathbf{v}_{\text{net},d} =
% \begin{bmatrix}
% v_{t,d} \\[4pt]
% v_{z,d}
% \end{bmatrix},
% \]
% where $v_{t,d}$ is the tangential velocity (across) and $v_{z,d}$ is the vertical velocity (up/down).  
% Radial motion $v_{r,d}$ is controlled independently to maintain the desired standoff distance.

% \subsection{Net Normal From Perception, but PID in $(y,z)$ Is Retained}

% In the real system, the outward net-normal vector $\mathbf{n}_\text{net}$ is estimated from the 
% camera-based perception system. This normal is used to:
% \begin{itemize}[noitemsep]
%     \item regulate the standoff in $x_\text{net}$,
%     \item align the yaw and pitch so the camera faces the net.
% \end{itemize}

% However, \emph{motion parallel to the net is still controlled using decoupled PID loops in 
% $(y_\text{net},z_\text{net})$}, as in the rectangular case. That is:

% \begin{itemize}[noitemsep]
%     \item When moving vertically (``up'' or ``down''), a PID controller holds the horizontal 
%           position $y_\text{net}$:
%           \[
%           \tau_y = -K_{p,y}(y_\text{net}-y_d) - K_{d,y}\dot{y}_\text{net} - K_{i,y}\int(y_\text{net}-y_d)\,dt.
%           \]

%     \item When moving horizontally (``across''), a PID controller holds the depth $z_\text{net}$:
%           \[
%           \tau_z = -K_{p,z}(z_\text{net}-z_d) - K_{d,z}\dot{z}_\text{net} - K_{i,z}\int(z_\text{net}-z_d)\,dt.
%           \]
% \end{itemize}

% Thus, although the net orientation is available through $\mathbf{n}_\text{net}$, the mode controller 
% implements the scanning motion exactly as in the rectangular case: \emph{hold one axis with PID while 
% moving along the other}.

% \subsection{Tangential Motion for Circular Nets}

% For circular nets, the horizontal progression is expressed as angular motion. The desired tangential 
% speed is kept constant:
% \[
% v_{t,d} = v_{\text{nominal}}.
% \]

% The angular tracking error is
% \[
% e_\varphi = \text{wrap}(\varphi - \varphi_d),
% \]
% and a PD or PI controller governs angular progress along the cage:
% \[
% \omega_d = -K_{p,\varphi} e_\varphi - K_{d,\varphi}\dot{e}_\varphi.
% \]

% This angular rate is converted to desired $y$/$z$ velocities using the cylindrical geometry, but the 
% cross-track axis is still regulated via PID as described above.

% \subsection{Radial (Standoff) Control}

% The standoff is controlled independently using the estimated normal direction. For circular nets,
% \[
% r \rightarrow r_d
% \]
% is enforced using a PI/PID controller:
% \[
% v_{r,d} = -K_{p,r} e_r - K_{d,r}\dot{e}_r - K_{i,r}\int e_r\,dt,
% \quad
% e_r = r - r_d.
% \]

% \subsection{Mode Switching}

% The same scanning pattern is used for both rectangular and circular nets:
% \[
% \text{down} \rightarrow \text{across} \rightarrow \text{up} \rightarrow \text{across}.
% \]

% For circular nets, the ``across'' condition becomes:
% \[
% |\text{wrap}(\varphi - \varphi_d)| < \epsilon_\varphi,
% \]
% while ``up'' and ``down'' transitions use the same depth thresholds as before.

% \subsection{Combined Net-Relative Command}

% The complete desired motion in the net frame is
% \[
% \mathbf{v}_{\text{net},d}
% =
% \begin{bmatrix}
% v_{r,d} \\[4pt]
% v_{t,d} \\[4pt]
% v_{z,d}
% \end{bmatrix},
% \]
% where $(v_{t,d},v_{z,d})$ follow the scanning pattern and the cross-track axis is always held by PID.

% This ensures that the up–across–down–across pattern is executed relative to the net geometry, even 
% for circular nets, while preserving the simple and robust decoupled PID structure of the original 
% mode controller.
% \subsection{Vertical Scanning Without Sideways Body Motion}

% During the downward scanning mode, the objective is that the vehicle moves as a straight line 
% in its \emph{own} body frame: it should descend with a constant vertical speed and experience 
% no sideways motion. At the same time, the global net-relative position in $y_\text{net}$ is 
% allowed to change due to currents or slow motion of the cage.

% Let $[u_b, v_b, w_b]^T$ denote the body-frame translational velocities in surge, sway, and heave,
% respectively. In the downward scan mode we impose the following velocity-level objectives:
% \[
% v_b \rightarrow 0,
% \qquad
% w_b \rightarrow w_d > 0,
% \]
% where $w_d$ is the desired constant scan speed.

% The sway motion is then controlled directly in the body frame by a PI/PID law on $v_b$:
% \[
% e_v = v_b - 0,
% \qquad
% \tau_2 = -K_{p,v}\, e_v - K_{d,v}\, \dot{e}_v - K_{i,v} \int e_v \, dt.
% \]
% This integral action on the body-frame sway velocity compensates for steady currents and 
% thruster asymmetries, and ensures that the vehicle does not drift sideways in its own frame 
% while scanning down the net.

% The vertical motion is controlled independently by regulating the body-frame heave velocity $w_b$ 
% towards the desired scan speed $w_d$:
% \[
% e_w = w_b - w_d,
% \qquad
% \tau_3 = -K_{p,w}\, e_w - K_{d,w}\, \dot{e}_w - K_{i,w} \int e_w \, dt.
% \]

% With this formulation, the vehicle trajectories in the global $(y_\text{net}, z_\text{net})$–plane 
% are not constrained to follow a fixed $y_\text{net}$ during the descent. Instead, the vehicle 
% follows a straight line in its body frame, while the net-relative position may shift slowly due 
% to environmental disturbances. At the bottom corner, the mode controller updates the reference 
% for the next horizontal segment (for example by snapping to the nearest path corner), thereby 
% re-synchronizing the scan pattern with the net.

% \section{Conditions for Straight-Line Net-Relative Descent}

% The downward scanning strategy introduced in this work commands the AUV to maintain
% \[
% v_b = 0, \qquad w_b = w_d > 0,
% \]
% in the body frame while keeping the vehicle perfectly aligned with the net normal. 
% In this section, we show mathematically that under these conditions the vehicle will 
% always follow a straight vertical line relative to the net, regardless of slow cage motion 
% or environmental drift, as long as the alignment condition is maintained.

% \subsection{Kinematic Relationship Between Body-Frame and Net-Frame Velocities}

% Let $\mathbf{v}_b = [u_b, v_b, w_b]^T$ denote the velocity in the body frame, and let 
% $R_{b}^{n}$ denote the rotation matrix from body to net frame. Then the net-relative velocity is
% \[
% \mathbf{v}_{\text{net}} = R_{b}^{n} \, \mathbf{v}_{b}.
% \]
% Since only motion in the $(y_\text{net},z_\text{net})$ plane is relevant during scanning, we project
% \[
% \mathbf{v}_{\parallel,\text{net}} 
% =
% \begin{bmatrix}
% y_\text{net}' \\[4pt]
% z_\text{net}'
% \end{bmatrix}
% =
% \begin{bmatrix}
% \mathbf{e}_{y_\text{net}}^T \\[4pt]
% \mathbf{e}_{z_\text{net}}^T
% \end{bmatrix}
% R_{b}^{n}\mathbf{v}_{b}.
% \]

% \subsection{Perfect Alignment With the Net}

% Perfect alignment means that the body $x$-axis is colinear with the net normal:
% \[
% \mathbf{e}_{x_b} \parallel \mathbf{n}_{\text{net}}.
% \]
% Equivalently,
% \[
% R_{b}^{n} =
% \begin{bmatrix}
% \mathbf{n}_{\text{net}} & \mathbf{t}_{\text{net}} & \mathbf{e}_{z_\text{net}}
% \end{bmatrix},
% \]
% where $\mathbf{t}_{\text{net}}$ is the horizontal tangent direction of the net.

% Under perfect alignment, the body-frame sway axis $\mathbf{e}_{y_b}$ lies exactly in the net plane.

% This implies:
% \[
% \mathbf{e}_{y_\text{net}}^T R_{b}^{n} = \mathbf{e}_{y_b}^T,
% \qquad
% \mathbf{e}_{z_\text{net}}^T R_{b}^{n} = \mathbf{e}_{z_b}^T,
% \]
% meaning the body-frame $v_b$ and $w_b$ map directly into the net's $y_\text{net}$ and $z_\text{net}$ 
% directions.

% \subsection{Straight-Line Descent Condition}

% With $v_b = 0$ enforced by a PI/PID controller, the velocity projected onto the net becomes:
% \[
% y_\text{net}' = \mathbf{e}_{y_\text{net}}^T R_{b}^{n}\mathbf{v}_b
% = \mathbf{e}_{y_b}^T \mathbf{v}_b
% = v_b
% = 0.
% \]
% Thus:
% \[
% y_\text{net}' = 0.
% \]

% Meanwhile, the vertical velocity is
% \[
% z_\text{net}' = \mathbf{e}_{z_\text{net}}^T R_{b}^{n}\mathbf{v}_b
% = \mathbf{e}_{z_b}^T \mathbf{v}_b
% = w_b
% = w_d.
% \]

% Hence, under perfect alignment:
% \[
% \begin{bmatrix}
% y_\text{net}' \\[4pt]
% z_\text{net}'
% \end{bmatrix}
% =
% \begin{bmatrix}
% 0 \\[4pt]
% w_d
% \end{bmatrix},
% \]
% which is the equation of a straight vertical line in the net frame.

% \subsection{Influence of Cage Drift and Currents}

% Even if the global frame moves due to currents or slow displacements of the cage, the net-relative 
% motion remains unchanged. Let the cage (and thus the net frame) move with velocity $\mathbf{v}_{c}$ 
% in NED. The net-relative velocity is
% \[
% \mathbf{v}_{\text{net}} = R_{b}^{n}\mathbf{v}_{b} - \mathbf{v}_{c}.
% \]

% Since the cage motion $\mathbf{v}_{c}$ affects both the vehicle and the net frame equally, it cancels 
% out in the relative motion:

% - the commanded $v_b = 0$ ensures no sideways drift \emph{relative to the net},  
% - the vertical scan speed $w_d$ is preserved relative to the net,  
% - global $y$ may drift but $y_\text{net}$ does not,  
% - mode switching based on $z_\text{net}$ still occurs correctly.

% Thus the vehicle remains on a straight vertical line in the net frame even if the cage moves.

% \subsection{Stability of the Zero-Sway Condition}

% The sway velocity control law
% \[
% \tau_y = -K_{p,v}\, v_b - K_{d,v}\, \dot{v}_b - K_{i,v} \int v_b \, dt
% \]
% drives $v_b \rightarrow 0$ exponentially provided the closed-loop dynamics are stable and the 
% integrator compensates for steady-state disturbances.

% Because the mapping between $v_b$ and $y_\text{net}'$ is linear and invertible under alignment,
% \[
% v_b = 0 \quad\Longleftrightarrow\quad y_\text{net}' = 0,
% \]
% the zero-sway condition is equivalent to zero sideways motion in the net frame.

% Therefore, the cross-track error remains identically zero during the descent.

% \subsection{Conclusion}

% If the AUV maintains perfect alignment with the net normal, then:
% \begin{enumerate}
%     \item enforcing $v_b = 0$ guarantees zero lateral motion in the net frame,
%     \item enforcing $w_b = w_d$ guarantees constant vertical scan speed in the net frame,
%     \item global drift in NED does not affect net-relative motion,
%     \item the trajectory in the $(y_\text{net},z_\text{net})$ plane is a straight line.
% \end{enumerate}

% Thus the proposed control structure achieves a provably straight vertical descent relative to the 
% net, even under environmental disturbances, provided alignment with the net normal is maintained.

% \subsection{Practical Limitations: Net Motion Is Not Observable From IMU Alone}

% The analysis above shows that, under perfect alignment and in the absence of relative motion 
% between the net frame and the inertial frame, enforcing
% \[
% v_b = 0, \qquad w_b = w_d
% \]
% leads to a straight-line descent in the $(y_\text{net}, z_\text{net})$–plane. However, this result 
% relies on an implicit assumption: that we know the motion of the net relative to the inertial frame, 
% or that this motion is negligible on the time scale of a single vertical scan.

% In practice, the AUV is equipped with an IMU (and possibly a DVL), which measure:
% \begin{itemize}[noitemsep]
%     \item angular rates and specific forces in the body frame,
%     \item and, after integration and aiding, the \emph{vehicle's} velocity and pose in an inertial frame 
%           (or relative to the seabed or water mass).
% \end{itemize}
% These sensors do \emph{not} directly provide the motion of the net or the cage. As a consequence, the 
% decomposition
% \[
% \mathbf{v}_{\text{net}} = R_b^n \mathbf{v}_b - \mathbf{v}_c,
% \]
% where $\mathbf{v}_c$ is the cage/net velocity, is not observable from the IMU alone: both the AUV and 
% the net may move under the influence of currents, and the IMU only sees the AUV motion.

% This means that the controller cannot \emph{separate} how much of the measured motion comes from the 
% vehicle or from the net. Therefore, the straight-line property derived above is a property of the 
% \emph{relative kinematics}, not something that can be guaranteed purely from IMU data without 
% additional information about the net.
% \subsection{Role of Perception in Net-Relative Alignment}

% To achieve net-relative motion in practice, the system relies on the perception module, which estimates:
% \begin{itemize}[noitemsep]
%     \item the net normal direction $\mathbf{n}_\text{net}$,
%     \item and, in some configurations, the approximate pose of the net relative to the vehicle.
% \end{itemize}

% This information is used to:
% \begin{itemize}[noitemsep]
%     \item align the AUV such that the body $x$–axis is approximately colinear with $\mathbf{n}_\text{net}$,
%     \item update the estimate of the net-relative position, especially at the top and bottom corners.
% \end{itemize}

% In the current implementation, it is assumed that:
% \begin{enumerate}
%     \item the net motion during a single vertical scan is slow compared to the commanded vertical 
%           scan speed $w_d$,
%     \item the perception system can periodically correct the net-relative position (for example when 
%           detecting the top or bottom of the net).
% \end{enumerate}

% Under these assumptions, enforcing zero body-frame sway $v_b \approx 0$ and a constant heave velocity 
% $w_b \approx w_d$ leads to an approximately straight scan line in the net frame, and any residual 
% net-relative drift can be corrected at the corners when updating the reference for the next segment.

% \subsection{Global Position Is Not Required}

% The inspection mission is formulated entirely in the net frame. The objective is not to follow a 
% trajectory in global NED coordinates, but to move along the net surface until the vehicle returns 
% to the starting location on the net. For this reason, absolute position estimates are unnecessary. 
% The only variables that must be controlled are those that describe motion \emph{relative to the net}.

% Let $(y_\text{net}, z_\text{net})$ denote coordinates along the net, and let $x_\text{net}$ be the 
% standoff direction. The scanning pattern is the repeated sequence
% \[
% \text{down} \rightarrow \text{across} \rightarrow \text{up} \rightarrow \text{across},
% \]
% and the mission is completed when the vehicle detects that it has reached the same angular position 
% or the same local net feature from which it started.

% However, because the IMU only measures the motion of the AUV and not the motion of the net, the 
% relative drift between the cage and the vehicle is not observable from inertial sensors alone. 
% Therefore the controller cannot determine how much of the measured motion originates from AUV 
% movement and how much originates from slow deformation or drift of the net.

% To ensure full coverage despite this limitation, the method relies on two assumptions:
% \begin{enumerate}
%     \item The net motion relative to the water is \emph{slow-varying} during a single vertical scan.
%     \item The horizontal step size between scan lines is chosen conservatively with intentional 
%           \emph{overlap}, so that small unmeasured net drift does not leave gaps in the inspection.
% \end{enumerate}

% Under these assumptions, controlling the body-frame sway velocity to zero during descent and 
% maintaining a constant vertical scan speed ensures that each scan line follows an approximately 
% straight path in the net frame. Any residual net-relative drift between scan lines is compensated 
% by the planned overlap in the inspection pattern.

% Because the final termination condition is defined in terms of returning to the initial location 
% on the net (e.g., by detecting the same angular position or the same top corner), global drift is 
% not required to be known or compensated. Net-relative sensing and slow, bounded drift are sufficient 
% for successful completion of the mission.
\section{Net-Relative Motion and Circular Nets}

In this work, all guidance and control are formulated \textbf{relative to the net}, rather than the 
global NED frame. The mission objective is not to follow an absolute trajectory in Earth-fixed 
coordinates, but to traverse the entire net surface in a systematic pattern. The inspection is 
complete once the vehicle returns to its starting location on the net. For this reason, global 
position is neither required nor used.

The coverage behavior is encoded directly in the \emph{mode controller} (see Chapter~\ref{ch:modecontroller}), 
which switches between vertical and horizontal motion modes and regulates standoff and orientation.

\subsection{Net Frame Definition}

To express all motion relative to the net, we define a net-fixed coordinate frame 
$\{\mathcal{N}_\text{net}\}$:
\begin{itemize}[noitemsep]
    \item $x_\text{net}$: outward normal from the net (standoff direction),
    \item $y_\text{net}$: horizontal direction along the net surface,
    \item $z_\text{net}$: vertical direction along the net (positive downward).
\end{itemize}

In the simplified planar simulation used in this thesis, the net frame is aligned with NED:
\[
x_\text{net} \parallel x_\text{NED},
\qquad 
y_\text{net} \parallel y_\text{NED},
\qquad 
z_\text{net} \parallel z_\text{NED},
\]
but all control laws are interpreted as \emph{net-relative}: the vehicle regulates its distance to 
the net in $x_\text{net}$ and performs the scanning motion in the $(y_\text{net},z_\text{net})$–plane.

\subsection{Extension to Circular Nets}

For a circular cage, the net is modeled as a vertical cylinder and the net-relative position is 
parameterized by
\[
(r,\,\varphi,\,z_\text{net}),
\]
where
\begin{itemize}[noitemsep]
    \item $r$ is the radial distance to the net surface (standoff),
    \item $\varphi$ is the angular coordinate around the cage (horizontal progression),
    \item $z_\text{net}$ is the vertical coordinate along the net.
\end{itemize}

The scanning pattern
\[
\text{down} \rightarrow \text{across} \rightarrow \text{up} \rightarrow \text{across}
\]
is preserved by interpreting:
\begin{itemize}[noitemsep]
    \item ``up/down''  $\rightarrow$ motion in $z_\text{net}$,
    \item ``across''   $\rightarrow$ motion in $\varphi$ (replacing linear $y_\text{net}$),
    \item standoff     $\rightarrow$ radial control $r \rightarrow r_d$.
\end{itemize}

The horizontal step size $SW$ in the planar case becomes an angular increment $\Delta\varphi$ for 
circular nets, such that each horizontal segment moves the vehicle from $\varphi_d$ to 
$\varphi_d + \Delta\varphi$.

\subsection{Net-Relative Mode Controller}

The mode controller (Chapter~\ref{ch:modecontroller}) defines four discrete motion modes:
\begin{enumerate}
    \item Mode 0: vertical descent along the net,
    \item Mode 1: vertical ascent along the net,
    \item Mode 2: horizontal motion at the bottom edge,
    \item Mode 3: horizontal motion at the top edge.
\end{enumerate}

At any time, exactly one mode is active. Each mode specifies:
\begin{itemize}[noitemsep]
    \item which axis is used for progression (vertical or horizontal),
    \item which axis is held at a reference using PID control,
    \item how the standoff and yaw are regulated.
\end{itemize}

For planar nets, horizontal motion is expressed directly in $y_\text{net}$; for circular nets, 
the same logic is applied to the angular coordinate $\varphi$.

\subsection{Vertical Scanning Without Sideways Body Motion}

During downward scanning (Mode~0), the objective is that the vehicle moves along a straight line 
in its \emph{body frame}: it should descend with a constant vertical speed and experience no 
sideways motion. The net-relative horizontal coordinate $y_\text{net}$ is allowed to drift slowly 
over the duration of one scan, and any residual error between scan lines is handled by overlap.

Let $[u_b, v_b, w_b]^T$ denote the body-frame translational velocities in surge, sway, and heave,
respectively. In the downward scan mode we impose:
\[
v_b \rightarrow 0,
\qquad
w_b \rightarrow w_d > 0,
\]
where $w_d$ is the desired constant scan speed.

The sway motion is controlled directly in the body frame by a PI/PID law:
\[
\tau_2 = -K_{p,v} v_b - K_{d,v}\dot{v}_b - K_{i,v}\!\int v_b\,dt,
\]
which suppresses steady sideways drift due to currents or thruster asymmetries.

The vertical motion is controlled independently by regulating $w_b$ towards $w_d$:
\[
\tau_3 = -K_{p,w}(w_b - w_d) 
         - K_{d,w}\dot{w}_b 
         - K_{i,w}\!\int (w_b - w_d)\, dt.
\]

With this formulation, the vehicle follows a straight line in its own frame during descent.  
The corresponding trajectory in $(y_\text{net},z_\text{net})$ will be approximately vertical as long 
as alignment with the net normal is maintained and net motion is slow.

At the bottom corner, the mode controller switches to Mode~2 and updates the reference for the 
next horizontal segment (for example snapping to the nearest path corner in $y_\text{net}$ or 
in $\varphi$), thereby re-synchronizing the scan pattern with the net.

\section{Conditions for Straight-Line Net-Relative Descent}

The downward scanning strategy described above commands the AUV to maintain
\[
v_b = 0, \qquad w_b = w_d > 0
\]
in the body frame while keeping the vehicle aligned with the net normal. Under these conditions, 
and assuming the net is locally static during a scan, the vehicle follows a straight vertical line 
relative to the net.

\subsection{Kinematic Relationship Between Body-Frame and Net-Frame Velocities}

Let $\mathbf{v}_b = [u_b, v_b, w_b]^T$ be the body-frame velocity, and let $R_{b}^{n}$ denote the 
rotation matrix from body to net frame. The net-relative velocity is
\[
\mathbf{v}_{\text{net}} = R_{b}^{n}\mathbf{v}_{b}.
\]
Projecting onto the net plane:
\[
\mathbf{v}_{\parallel,\text{net}} 
=
\begin{bmatrix}
y_\text{net}' \\[4pt]
z_\text{net}'
\end{bmatrix}
=
\begin{bmatrix}
\mathbf{e}_{y_\text{net}}^T \\[4pt]
\mathbf{e}_{z_\text{net}}^T
\end{bmatrix}
R_{b}^{n}\mathbf{v}_{b}.
\]

\subsection{Perfect Alignment With the Net}

Perfect alignment means that the body $x$-axis is colinear with the net normal:
\[
\mathbf{e}_{x_b} \parallel \mathbf{n}_\text{net}.
\]
Then $R_{b}^{n}$ can be written such that
\[
R_{b}^{n} =
\begin{bmatrix}
\mathbf{n}_\text{net} & \mathbf{t}_\text{net} & \mathbf{e}_{z_\text{net}}
\end{bmatrix},
\]
where $\mathbf{t}_\text{net}$ is the horizontal tangent and $\mathbf{e}_{z_\text{net}}$ the vertical net axis.

In this case,
\[
\mathbf{e}_{y_\text{net}}^T R_{b}^{n} = \mathbf{e}_{y_b}^T,
\qquad
\mathbf{e}_{z_\text{net}}^T R_{b}^{n} = \mathbf{e}_{z_b}^T,
\]
so $v_b$ and $w_b$ map directly into $y_\text{net}'$ and $z_\text{net}'$.

\subsection{Straight-Line Descent Condition}

With $v_b = 0$ enforced:
\[
y_\text{net}' 
= \mathbf{e}_{y_\text{net}}^T R_{b}^{n}\mathbf{v}_b
= \mathbf{e}_{y_b}^T \mathbf{v}_b
= v_b
= 0.
\]

Meanwhile:
\[
z_\text{net}' 
= \mathbf{e}_{z_\text{net}}^T R_{b}^{n}\mathbf{v}_b
= \mathbf{e}_{z_b}^T \mathbf{v}_b
= w_b
= w_d.
\]

Hence,
\[
\begin{bmatrix}
y_\text{net}' \\[4pt]
z_\text{net}'
\end{bmatrix}
=
\begin{bmatrix}
0 \\[4pt]
w_d
\end{bmatrix},
\]
which is the equation of a straight vertical line in the net frame.

\subsection{Stability of the Zero-Sway Condition}

The sway controller
\[
\tau_y = -K_{p,v}v_b - K_{d,v}\dot{v}_b - K_{i,v}\int v_b\,dt
\]
drives $v_b \rightarrow 0$ provided the closed-loop dynamics are stable. Because the mapping between 
$v_b$ and $y_\text{net}'$ is linear and invertible under alignment,
\[
v_b = 0 \quad\Longleftrightarrow\quad y_\text{net}' = 0,
\]
zero-sway implies zero net-relative sideways motion during descent.

\section{Practical Limitations: Net Motion Is Not Observable From IMU Alone}

The analysis above assumes a static net frame during a single scan. In practice, the IMU (and DVL) 
measure the motion of the \emph{AUV}, not the motion of the net. The relative drift between cage 
and vehicle is therefore not observable from inertial sensors alone, and the method cannot separate 
AUV motion from net motion.

To ensure full coverage despite this limitation, the method relies on two assumptions:
\begin{enumerate}
    \item The net motion relative to the water is \emph{slow-varying} during a single vertical scan.
    \item The horizontal step size between scan lines is chosen conservatively with intentional 
          \emph{overlap}, so that small unmeasured net drift does not leave gaps in the inspection.
\end{enumerate}

Under these assumptions, controlling $v_b \approx 0$ and $w_b \approx w_d$ yields an approximately 
straight scan line in the net frame, and residual drift between scan lines is compensated by overlap.

\section{Role of Perception in Net-Relative Alignment}

Net-relative operation depends on the perception module, which estimates:
\begin{itemize}[noitemsep]
    \item the net normal $\mathbf{n}_\text{net}$,
    \item characteristic features such as the top and bottom of the net or strong corner points.
\end{itemize}

These are used to:
\begin{itemize}[noitemsep]
    \item align the AUV with the net so that the body $x$-axis follows $\mathbf{n}_\text{net}$,
    \item correct the net-relative position at top and bottom, and when closing the loop.
\end{itemize}

\section{Global Position Is Not Required}

The mission is defined purely in terms of net-relative coordinates. The scanning pattern is the 
repeated sequence
\[
\text{down} \rightarrow \text{across} \rightarrow \text{up} \rightarrow \text{across},
\]
and the mission ends when the vehicle detects that it has returned to its starting location on the net 
(e.g.\ same angular position or same top corner). Global drift in NED is neither measured nor required 
to be compensated; slow, bounded net motion and net-relative sensing are sufficient for inspection.

\subsection{Net-Relative Control During Horizontal (Across) Motion}

During across-segments (Modes~2 and~3), the AUV moves horizontally along the net surface to reach 
the next vertical scan line. Unlike the vertical segments, where the body-frame sway velocity is 
forced to zero, the across-direction requires the vehicle to generate a controlled, constant 
horizontal velocity while holding depth. The same logic applies to both planar and circular nets.

\paragraph{Planar Nets.}
For planar nets, the across-direction is the $y_{\text{net}}$–axis. The mode controller chooses a 
desired horizontal reference velocity:
\[
v_{\mathrm{ref}} = 
\max\!\left(v_{\min},\; \alpha v_d\right),
\qquad
\alpha = 
\min\!\left(1,\; 
\frac{|y_d + SW - y_{\text{net}}|}{d_{\text{corner}}}
\right),
\]
where $SW$ is the horizontal step width between vertical scan lines. The saturation through 
$\alpha$ ensures smooth slowdown when approaching the next corner.

Horizontal motion is enforced through a velocity-tracking controller:
\[
\tau_2 = 
- K_{p,v}(v - v_{\mathrm{ref}})
- K_{d,v}\dot{v},
\]
where $v$ is the sway velocity in the NED/net frame.  
Depth is held using a PID controller in the vertical direction:
\[
\tau_3 = 
- K_{p,z}(z_{\text{net}} - z_d)
- K_{d,z}w
- K_{i,z} \!\int (z_{\text{net}} - z_d)\,dt,
\]
with $z_d = z_{\text{bot}}$ in Mode~2 and $z_d = z_{\text{top}}$ in Mode~3.

\paragraph{Circular Nets.}
For circular nets, the across-direction becomes the angular coordinate $\varphi$. The same control 
principle applies, but with the reference expressed in angular space:
\[
\varphi_d \leftarrow \varphi_d + \Delta\varphi.
\]

The corresponding tangential velocity is
\[
v_{t,\mathrm{ref}} = R\,\dot{\varphi}_{\mathrm{ref}},
\]
where $R$ is the local net radius.  
The angular error is
\[
e_\varphi = \text{wrap}(\varphi - \varphi_d),
\]
and is regulated by a PI/PD controller:
\[
\dot{\varphi}_{\mathrm{cmd}} 
= -K_{p,\varphi}e_\varphi - K_{d,\varphi}\dot{e}_\varphi.
\]

This angular rate is mapped into Cartesian sway/heave components via the tangent vector
\[
\mathbf{t}(\varphi) =
\begin{bmatrix}
-\sin \varphi \\[4pt]
\cos \varphi
\end{bmatrix},
\]
so the desired horizontal motion becomes
\[
\mathbf{v}_{t,d} = 
v_{t,\mathrm{ref}}\,\mathbf{t}(\varphi).
\]

Vertical motion is again held with a PID controller identical to the planar case.

\paragraph{Mode Transition.}
Across-motion ends when the commanded horizontal displacement is achieved:
\[
|y_{\text{net}} - (y_d + SW)| < \epsilon_y
\qquad \text{(planar net)},
\]
or
\[
|\text{wrap}(\varphi - \varphi_d)| < \epsilon_\varphi
\qquad \text{(circular net)}.
\]

The controller then resets any accumulated integral error in the held axis and switches back to a 
vertical mode (Mode~0 or Mode~1). This ensures that integral contributions from the across-segment 
do not pollute the vertical scan behavior.

\paragraph{Summary.}
Across-direction control combines:
\begin{itemize}[noitemsep]
    \item constant horizontal velocity tracking,
    \item slow corner-approach damping,
    \item depth-holding PID control,
    \item and a clean handoff to the next vertical segment.
\end{itemize}
This structure ensures that the scanning pattern progresses smoothly and that each vertical scan 
starts at the correct lateral offset—even when subject to slow, unobservable net motion—because 
horizontal stepping is re-synchronized at each mode transition.

\subsection{Net-Relative Across-Direction Control Using Instantaneous Net Normal}

When moving horizontally along the net (Modes~2 and~3), the controller does not rely on a global
$y$-axis or on a fixed geometric tangent direction. Instead, the AUV moves \emph{relative to the 
current shape of the net}. At every control cycle, the perception system provides an estimate of the 
outward net normal vector
\[
\mathbf{n}_{\text{net}}(t),
\]
which may vary slowly due to net deformation or cage motion.

\paragraph{Constructing the Tangent Direction.}
Given the normal vector $\mathbf{n}_{\text{net}}$, the tangent direction of allowed motion along 
the net is obtained by projecting the desired global horizontal direction onto the net surface. 
A unique tangent can be constructed using the cross product with the net’s vertical axis:
\[
\mathbf{t}_{\text{net}} 
= \frac{\mathbf{e}_{z} \times \mathbf{n}_{\text{net}}}
       {\|\mathbf{e}_{z} \times \mathbf{n}_{\text{net}}\|},
\]
where $\mathbf{e}_{z}$ is the downward vertical axis of the net frame.  
This produces a unit vector lying in the net surface and perpendicular to the net normal.

\paragraph{Feed-Forward Tangential Motion.}
Instead of regulating position along the across-direction, horizontal motion is generated purely by 
\emph{feedforward thrust} along the instantaneous tangent:
\[
\mathbf{v}_{\parallel,d}
= v_{t,d}\, \mathbf{t}_{\text{net}},
\]
where $v_{t,d}$ is the nominal horizontal scanning speed.  
In practice, the horizontal command is implemented by applying a constant desired sway/heave 
velocity (or thrust) in the direction of $\mathbf{t}_{\text{net}}$.

Thus, the AUV’s across-motion follows the \emph{local geometry} of the net surface.
If the net bulges or deforms slightly, the horizontal motion bends correspondingly, ensuring that 
coverage remains net-relative instead of world-relative.

\paragraph{Depth Holding During Across Motion.}
While moving along $\mathbf{t}_{\text{net}}$, the depth $z_{\text{net}}$ is held constant by a PID
controller:
\[
\tau_3 = 
- K_{p,z}(z_{\text{net}} - z_d)
- K_{d,z}\dot{z}_{\text{net}}
- K_{i,z} \!\int (z_{\text{net}} - z_d)\,dt,
\]
with $z_d = z_{\text{top}}$ in Mode~3 and $z_d = z_{\text{bot}}$ in Mode~2.

\paragraph{Alignment With the Net Normal.}
To ensure the tangential direction is meaningful, the AUV continuously aligns its body $x$-axis to 
the net normal:
\[
\mathbf{e}_{x_b} \parallel \mathbf{n}_{\text{net}}.
\]
This is achieved by a yaw controller that regulates the heading so the camera always faces the net.
As long as the alignment error remains small, the tangent $\mathbf{t}_{\text{net}}$ directly maps 
into body-frame sway/heave commands.

\paragraph{Mode Completion.}
Across-motion ends when the vehicle has moved one predefined step width along the net.  
Since net motion is not directly observable from IMU or DVL, the controller does not rely on 
global coordinates but instead uses:
\begin{itemize}[noitemsep]
    \item detection of the next scanline boundary (planar net), or
    \item monitoring angular progress $\varphi$ (circular net), or
    \item detection of a landmark in the perception system.
\end{itemize}
Once the step is complete, the mode controller transitions to the next vertical mode and resets 
the depth integrator.

\paragraph{Summary.}
Across-direction control is fully net-relative:
\begin{itemize}[noitemsep]
    \item the tangent direction is recomputed at every cycle from the measured net normal,
    \item the AUV aligns its body to the net and applies a constant feedforward thrust along the tangent,
    \item depth is held by PID control,
    \item mode transitions are based on net-relative measurements, not global motion.
\end{itemize}
This guarantees that the horizontal motion traces the \emph{actual} net surface, even when the cage 
moves or deforms slowly.

\subsection{IMU Accuracy and the Feasibility of Net-Relative Across Motion}

A key motivation for executing horizontal (across) motion relative to the \emph{local net geometry}
rather than a global reference frame is the difference in accuracy between short-term inertial
measurements and long-term absolute position estimation. An IMU provides high-bandwidth measurements
of angular rates and accelerations, which integrate to velocity with relatively low short-term
drift. Over time scales of a few seconds, and over spatial scales of only a few meters, the 
resulting velocity and displacement estimates are significantly more accurate than global position 
estimates derived from dead reckoning.

In the across-direction, each horizontal step typically spans only 
$\mathcal{O}(2$--$4\,\mathrm{m})$. Over such distances, the IMU can measure changes in body-frame
velocity and orientation with high fidelity:
\begin{itemize}[noitemsep]
    \item short-term integration drift is very small,
    \item the relative change in heading is accurately observable,
    \item the body-frame sway and heave velocities are measured directly or aided by DVL.
\end{itemize}

Thus, even though the IMU cannot provide an accurate \emph{global} position, it can reliably
indicate that the vehicle has moved a few meters along a commanded direction. This is precisely the 
scale of the across-step between vertical scan lines.

Because the tangent direction $\mathbf{t}_{\text{net}}$ is recomputed at every control cycle from the 
instantaneous net normal, the vehicle does not depend on knowledge of its global pose. The controller
simply applies:
\[
\mathbf{v}_{\parallel,d} = v_{t,d}\,\mathbf{t}_{\text{net}},
\]
and the IMU/DVL measurements confirm that the vehicle has progressed the required short distance. 
Any residual drift due to slow net motion is corrected when switching back into a vertical mode, and
the deliberate overlap between scan lines ensures that coverage gaps do not occur.

In summary, the across-direction control strategy leverages the fact that:
\begin{enumerate}
    \item IMU-based velocity integration is reliable over short time horizons,
    \item across-steps are intentionally kept short (a few meters),
    \item net-relative alignment eliminates the need for absolute global position, and
    \item overlap in the scanning pattern compensates for small unobservable net motion.
\end{enumerate}

This makes net-relative across motion both feasible and robust, even in the absence of accurate
global navigation.

\subsection{Using Global IMU-Based Position Estimates in a Relative Manner}

Although the inspection mission is formulated in the net frame and does not rely on global
navigation, the AUV still uses the pose estimate provided by the IMU (and DVL) to infer how far it 
has moved between control updates. Importantly, this information is used in a \emph{relative}, 
short-term sense rather than as an absolute global position.

The IMU provides accurate, high-rate estimates of velocity and orientation, and when aided by the 
DVL, it also yields drift-limited estimates of position over short time horizons. While global 
position from dead reckoning accumulates error rapidly over long distances, the short horizontal 
across-steps in the coverage pattern are only on the order of
\[
2\text{--}4~\mathrm{m},
\]
which the IMU/DVL system can measure reliably.

Thus, the vehicle uses its global estimated position increment
\[
\Delta \mathbf{p} = \mathbf{p}(t) - \mathbf{p}(t_0)
\]
only to determine whether it has moved the required \emph{relative} distance along the commanded
tangent direction $\mathbf{t}_{\text{net}}$. No absolute global accuracy is required.

This relative use of the global estimate has several advantages:
\begin{itemize}[noitemsep]
    \item Short-term IMU–DVL integration drift is very small over a few meters.
    \item The commanded across-motion is executed according to the \emph{local} net geometry.
    \item The AUV is not required to know whether the cage itself is drifting.
    \item Any residual horizontal drift is compensated through intentional overlap between scan lines.
\end{itemize}

Formally, the vehicle checks progress using the projection:
\[
s(t) = \mathbf{t}_{\text{net}}^T \Delta\mathbf{p},
\]
and the across-step is complete when $s(t)$ reaches a prescribed length
$SW$ (planar net) or an angular increment $\Delta\varphi$ (circular net). This means that even though
the measurement $\mathbf{p}(t)$ is in a global frame, it is used only to evaluate \emph{relative 
motion along the net}, making the method insensitive to global drift.

In this way, the IMU/DVL-based global estimate acts as a reliable short-term odometer rather than an 
absolute navigator, aligning naturally with the net-relative nature of the scanning task.

\subsection{Detecting Completion of the Across Motion}

During horizontal (across) motion, the vehicle must determine when it has moved the required
distance to begin the next vertical scan. Because the absolute global position is not required for
the mission, the AUV uses only the \emph{relative} change in its estimated global position to detect
across-progress.

Let $\mathbf{p}(t)$ be the global position estimate (IMU/DVL-aided), and let $t_0$ denote the time
when the across-segment begins. The relative displacement is
\[
\Delta\mathbf{p}(t) = \mathbf{p}(t) - \mathbf{p}(t_0).
\]

The tangent direction of the net at the current position, obtained from the estimated net normal
$\mathbf{n}_{\text{net}}$, is
\[
\mathbf{t}_{\text{net}}
= 
\frac{\mathbf{e}_{z} \times \mathbf{n}_{\text{net}}}
{\|\mathbf{e}_{z} \times \mathbf{n}_{\text{net}}\|},
\]
which lies in the net surface and represents the “across’’ direction.

The component of the displacement along the net is then
\[
s(t) = \mathbf{t}_{\text{net}}^{T}\,\Delta\mathbf{p}(t),
\]
which represents how far the AUV has moved \emph{along the net}, regardless of any drift normal to
the net.

\paragraph{Planar nets.}
For a planar net, each across-step has a fixed linear width $SW$. The across motion is complete when
\[
s(t) \geq SW.
\]

\paragraph{Circular nets.}
For circular cages, the across motion is defined by a target angular increment $\Delta\varphi$.
Using the cage radius $R$, the linear step length is
\[
SW = R\,\Delta\varphi,
\]
and the same condition $s(t) \geq SW$ is used to detect completion.

\paragraph{Robustness.}
This method has several advantages:
\begin{itemize}[noitemsep]
    \item it uses only short-term IMU/DVL motion, which is accurate over 2–4 m,
    \item it does not require absolute global position,
    \item it is naturally aligned with the local net geometry,
    \item it isolates the horizontal progress even if the net or cage drifts.
\end{itemize}

Thus, the vehicle reliably detects the completion of each across-segment using purely relative
motion projected onto the net tangent direction.
