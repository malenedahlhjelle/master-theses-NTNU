

\chapter{Mode-Based Net-Following Control} In this chapter we present and analyze a mode-based hybrid controller used for autonomous inspection of cylindrical aquaculture nets using an underwater vehicle (AUV/ROV). The controller organizes the motion into four modes: moving \emph{down} along a vertical stripe, \emph{up} along a vertical stripe, \emph{bottom horizontal} motion, and \emph{top horizontal} motion. Together these modes generate a systematic ``lawnmower''-like pattern around the net. We prove fundamental properties about the stability of the closed-loop behavior and justify the switching logic. 
\section{Controller Overview} Let ``$\eta = (x,y,z,\phi,\theta,\psi)$'' denote the pose in earth-fixed coordinates and Euler angles, and let ``$\nu = (u,v,w,p,q,r)$'' denote the body velocity vector. The control input is the generalized force vector ``$\tau = (\tau_1,...,\tau_6)$``. The net surface is parameterized by an inward-pointing surface normal ``$n_\mathrm{net}$`` and a scalar ``$s_\mathrm{clear}$`` describing the signed clearance along this normal, where positive means the vehicle is inside, zero means exactly on the net, and negative means outside. The controller uses persistent internal state variables to represent the current stripe angle, the next stripe angle, vertical mode logic, and integral terms used for PI control in several channels. The mode variable is an integer ``$m \in \{0,1,2,3\}$`` with the meaning: \begin{itemize} \item $m=0$: vertical motion down along stripe $\theta=\theta_\mathrm{curr}$, \item $m=1$: vertical motion up, \item $m=2$: horizontal motion along the bottom toward $\theta_\mathrm{next}$, \item $m=3$: horizontal motion along the top. \end{itemize} Each mode implements a feedback law that is smooth within the mode. Switching occurs when the AUV reaches the bottom, top, or the next stripe angle.
\section{Surge-Clearance Control} The first DOF regulates the inward clearance:\[ s_\mathrm{clear} \to x_d, \] where $x_d>0$ is the desired inward offset from the net. A PI controller with velocity damping is applied: \begin{equation} \tau_1 = -K_{p1}( s_\mathrm{clear} - x_d ) - K_{d1} u - K_{i1} \, \int (s_\mathrm{clear}-x_d)\,dt. \end{equation} This is a standard exponentially stabilizing PD--PI controller for regulating a scalar reference under bounded disturbances. Under the assumption that the surge dynamics are input-signal strictly passive (standard for underwater vehicles), the closed loop is input-to-state stable (ISS). The integral term ensures zero steady-state error. 
\section{Sway Angle Tracking in Vertical Modes} When moving vertically ($m=0$ or $m=1$), the controller attempts to keep the vehicle aligned with the stripe described by angle $\theta_\mathrm{curr}$. Define the stripe tracking error \begin{equation} e_\theta = \operatorname{ssa}(\theta_\mathrm{rel} - \theta_\mathrm{curr}), \end{equation} where $\operatorname{ssa}(\cdot)$ wraps angles to $(-\pi,\pi]$. The sway controller is \begin{equation} \tau_2 = -K_{p\theta} e_\theta - K_{d\theta} v - K_{i\theta} \int e_\theta \,dt. \end{equation} Again, under standard marine-vehicle passivity assumptions, this is a globally stable PI-damped controller with exponential convergence of $e_\theta$. 
\section{Vertical Velocity Control and Corner Slowdown} To move vertically with nominal rate $w_d$, the controller applies \begin{equation} \tau_3 = -K_{p3}(w - w_\mathrm{ref}), \end{equation} where the reference $w_\mathrm{ref}$ is filtered near the top/bottom boundaries with a corner slowdown factor \begin{equation} w_\mathrm{ref} = \operatorname{sign}(\text{direction}) \cdot \max( w_\mathrm{corner,min},\; w_d \, \alpha ),\quad \alpha = \min\left(1, \frac{\mathrm{dist}}{d_\mathrm{corner}}\right). \end{equation}  This ensures smooth deceleration before switching modes, avoiding Zeno behavior. The resulting subsystem is an overdamped first-order system that guarantees boundedness and smoothness. 

\section{Horizontal Modes and Stripe Updates} In bottom and top horizontal modes ($m=2,3$), the system moves tangentially around the net at desired_velocity $v_d$ using the sway controller \begin{equation} \tau_2 = -K_{p\theta}(v - v_d). \end{equation} Depth is maintained by a PI controller: \begin{equation} \tau_3 = -K_{p3,z}(z - z_d) - K_{d3} w - K_{i3} \int (z - z_d) \, dt. \end{equation} The next stripe angle $\theta_{\text{next}}$ is reached once ``$\theta_{\mathrm{rel}} > \theta_{\mathrm{next}} - \theta_{\mathrm{tol}}$``. Because $\theta_\mathrm{rel}$ is monotonic in the horizontal traversal, the condition is well-posed. Switching is thus deterministic and nonambiguous. \section{Attitude Alignment with the Net Surface} The desired body $x$-axis direction is the inward surface normal $n$. Using ZYX Euler angles, the direction of the body $x$-axis is \begin{equation} d(\phi,\theta,\psi) = (\cos\theta\cos\psi,~ \cos\theta\sin\psi,~ -\sin\theta). \end{equation} Equating $d=n$ yields desired attitudes \begin{equation} \theta_d = -\arcsin(n_z), \qquad \psi_d = \operatorname{atan2}(n_y,n_x). \end{equation} Errors are \begin{equation} e_\theta = \operatorname{ssa}(\theta-\theta_d), \qquad e_\psi = \operatorname{ssa}(\psi-\psi_d). \end{equation} The pitch and yaw controllers are \begin{align} \tau_5 &= -K_{p5} e_\theta - K_{d5} q - K_{i5}\!\int e_\theta dt,\\ \tau_6 &= -K_{p6} e_\psi - K_{d6} r - K_{i6}\!\int e_\psi dt. \end{align} The roll axis is damped to 0 using \begin{equation} \tau_4 = -K_{p4} \phi - K_{d4} p. \end{equation} Because the attitude dynamics are passive and the reference is constant in each mode, these PI-damped controllers guarantee exponential convergence modulo the Euler singularity set, which is never reached due to the physical geometry of a net. \section{Closed-Loop Hybrid Stability} The full system is a hybrid automaton with four continuous domains and three switching surfaces. Each continuous domain has a smooth control Lyapunov function (CLF): a weighted sum of squared tracking errors in\ $s_\mathrm{clear}$, $z$, tangential $v$, attitude errors, and integral states. At switching events, the CLF does not increase because: \begin{enumerate} \item Stripe updates reset only the stripe-integral state, which eliminates stored energy. \item Vertical transitions occur at physical boundaries ($z=0$ or $z=H$) where vertical velocity is already near zero due to corner slowdown. \item Other states ($u,v,p,q,r$) are continuous. \end{enumerate} Thus the hybrid system satisfies the standard dwell-time condition and has no Zeno executions. The CLF decreases strictly in each mode, hence the system is globally asymptotically stable with respect to the desired stripe-following orbit. \section{Conclusion} We have presented a complete description of the mode-based controller used for net inspection. Each mode is individually stable and smooth, while switching is triggered by geometric events that preserve Lyapunov stability. The resulting hybrid closed-loop system produces a predictable, robust stripe pattern that fully covers the net and maintains safe clearance, alignment, and depth. 
\end{verbatim}
