

This chapter describes the operational setting, net geometries, coordinate conventions, and the AUV model used throughout the project thesis. The purpose is to define the physical and modeling framework within which the presented methods are developed, tested and evaluated. 

\section{Operational Scenario}
The inspection mission takes place around a standard marine aquaculture net pen at surface level with varying depths down to 50m. The AUV operates in the water column surrounding the net, moving along its surface at a fixed stand-off distance. Depths range from the surface to the bottom of the cylindrical enclosure, and currents may act on the vehicle throughout the mission. It is desirable for the AUV to move in vertical segments as mooring lines provide estimation clues and previous trials within the company showed difficulties following the net structure horizontally because of inconsistent net shape also due to mooring lines.

The figure \ref{fig:deformation} from \cite{lopez2015volumeloss} below displays anticipated net deformation dependent on the sinker ring of the fish farm. It shows a significant amount of deformation and is one of the main issues this project thesis will address.
\begin{figure}
    \centering
    \includegraphics[width=0.5\linewidth]{image.png}
    \caption{Enter Caption}
    \label{fig:deformation}
\end{figure}

The vehicle receives pose, attitude, and depth estimates from its onboard estimator. It also receives the its radial position in the current hight level of the net. This requires that the estimator estimates the deformation of the net, which was shown to be possible in\cite{amundsen2025adaptive}. It will also assume discrete perception cues such as detection of the net top and bottom boundaries. These inputs are made available to the guidance system at a sufficiently high rate during normal circumstances to allow closed-loop operation. It is assumed that the estimator will have temporary stalls in providing these inputs. Cause by disturbances, drift and occlusion of the camera, which the guidance system must account for.

\section{Reference Frames}

All quantities are expressed in the North–East–Down (NED) frame unless stated otherwise. The
frame defines the spatial coordinates for vehicle pose and the directions of motion used by the low-level controllers.

We introduce a second frame \emph{netframe}. For the planar case, the net frame coincides with NED. For the cylindrical case, a local net-relative frame is introduced in which the tangential direction follows the perimeter of the pen, the vertical axis corresponds to depth, and the inward normal defines the desired camera-facing direction. This frame is used internally by the guidance system to express desired motion relative to the net surface.

The AUV control will be expressed in the body frame on the AUV, which corresponds to forwar-right-down. 

\section{Net Geometries}
Two static net geometries are used in development and simulation. The first is a planar surface which is aligned with the NED-frame. The second is a deformed cylindrical surface representing the full net pen shown in \ref{fig:net}. In the latter case the developed method is not allowed access to the full shape of the net-pen, only the information available at its current position, allowing for easy transfer to the real scenario.

\begin{figure}
    \centering
    \includegraphics[width=0.5\linewidth]{net.png}
    \caption{Enter Caption}
    \label{fig:net}
\end{figure}

\section{AUV Model}

The AUV dynamics are based on the standard 6-DOF marine vehicle model described in
Fossen~\cite{fossen2021handbook}. The full nonlinear equations of motion are

\begin{equation}
M_{RB}\dot{\boldsymbol{\nu}} + M_A \dot{\boldsymbol{\nu}} +
C_{RB}(\boldsymbol{\nu})\boldsymbol{\nu} +
C_A(\boldsymbol{\nu})\boldsymbol{\nu} +
D(\boldsymbol{\nu})\boldsymbol{\nu} +
G(\boldsymbol{\eta}) = \boldsymbol{\tau} + \boldsymbol{\tau_{disturbance}}, 
\label{eq:fossen-full}
\end{equation}

with kinematics

\begin{equation}
\dot{\boldsymbol{\eta}} = J(\boldsymbol{\eta}) \boldsymbol{\nu}.
\end{equation}

Here, $\boldsymbol{\eta} = [x, y, z, \phi, \theta, \psi]^T$ is the generalized position in the NED
frame, and $\boldsymbol{\nu} = [u, v, w, p, q, r]^T$ is the body-fixed velocity vector.
$M_{RB}$ and $M_A$ denote rigid-body and added-mass inertia,
$C_{RB}(\boldsymbol{\nu})$ and $C_A(\boldsymbol{\nu})$ the associated Coriolis and centripetal
matrices, $D(\boldsymbol{\nu})$ the nonlinear and linear hydrodynamic damping, and $G(\boldsymbol{\eta})$ the
restoring forces from gravity and buoyancy. The control input $\boldsymbol{\tau}$ represents forces
and moments generated by the thrusters.

\subsection*{Model Simplifications}

For high-fidelity simulation or control allocation, the full model in
(\ref{eq:fossen-full}) is necessary. However, for the purpose of developing and evaluating guidance
laws for slow, quasi-static inspection motion, several simplifications are justified:

\paragraph*{Neutral buoyancy.}
The vehicle is assumed neutrally buoyant, implying that gravity and buoyancy nearly cancel. The
restoring vector reduces to roll and pitch components only,
\[
G(\boldsymbol{\eta}) = [0, 0, 0, G_\phi, G_\theta, 0]^T,
\]
with negligible contributions in surge, sway, heave, and yaw. It is further assumed that the center
of buoyancy and the center of mass are collocated. Under this assumption, the restoring forces in
the translational degrees of freedom vanish, and only the roll and pitch moments remain. This allows
the restoring vector to retain the simplified form
\[
G(\boldsymbol{\eta}) =  0
\]

\paragraph*{Low-speed manoeuvring.}
Net inspection requires slow and smooth motion. At low velocities,
Coriolis and centripetal forces ($C_{RB}$, $C_A$) are small relative to drag and can be omitted
without affecting guidance behaviour.

\paragraph*{Weak hydrodynamic coupling.}
The AUV operates without aggressive accelerations, and cross-coupling terms in the added mass and
damping are minor. This allows both inertia and damping to be approximated as diagonal matrices:
\[
M = \mathrm{diag}(m_{11}, m_{22}, m_{33}, m_{44}, m_{55}, m_{66}), \qquad
D = \mathrm{diag}(d_{11}, d_{22}, d_{33}, d_{44}, d_{55}, d_{66}),
\]
which significantly reduces model complexity.

%\paragraph*{Passive roll and pitch stability.}
%The AUV's physical design ensures that roll remains small and naturally stabilised. These states are therefore not actively controlled and are treated as passive during guidance.

\paragraph*{Guidance-level abstraction.}
The purpose of this thesis is to design guidance and path planning methods to ensure complete inspection, not a full dynamic control system. Thus, the simplified dynamics model is sufficient to capture motion relevant to the guidance system. In reality the area behind the net is highly dynamic due to the of the net creating a wake, requiring advanced control methods out of the scope of this project.[some source]

\subsection*{Simplified Model Used in this Work}

Applying the above assumptions yields the reduced dynamic model

\begin{equation}
M\dot{\boldsymbol{\nu}} + D\boldsymbol{\nu} = \boldsymbol{\tau},
\end{equation}

with kinematics unchanged:

\begin{equation}
\dot{\boldsymbol{\eta}} = J(\boldsymbol{\eta})\boldsymbol{\nu}.
\end{equation}

This reduced representation preserves the essential surge, sway, heave, pitch, and yaw behaviour needed for path-following and inspection, while eliminating the complex hydrodynamic couplings not relevant to the slow and deliberate manoeuvres required during aquaculture net inspection.

\section{Key Assumptions}

\begin{itemize}
    \item Currents are assumed to be moderate and slowly varying.
    \item The vehicle’s localization is assumed accurate, except for occasional short-lived degradations.
    \item Net geometries used for planning and evaluation are idealised planar and deformed cylindrical surfaces.
    \item Boundary detections from perception are provided as discrete events.
\end{itemize}


\section{Requirements}

The requirements for the autonomous guidance system are shown in Table~\ref{table:req}. These
requirements specify the operational constraints, performance targets, and interface expectations used throughout the development and evaluation of the guidance methods.

% Your existing requirements table remains unchanged here.


