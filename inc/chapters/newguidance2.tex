\section{Guidance System}

This chapter presents the guidance strategy used to follow the inspection path along the
net surface. The key idea is that the vehicle moves with a fixed, user-defined desired speed
$U_{\text{des}}$ along the net. The path is defined in a local 2D path frame, and its tangent
vector is rotated into the net plane for execution. The local surface normal is used both for
orientation and for regulating the standoff distance. In this way, the system avoids steering
based directly on global positions, which may be inaccurate near the net.

\subsection{Coordinate Frames}

The vehicle uses the standard NED frame:
\[
x \text{ (north)}, \qquad y \text{ (east)}, \qquad z \text{ (down)}.
\]

The inspection path is defined in a 2D path frame $\{P\}$:
\[
p_P(s) =
\begin{bmatrix}
y_P(s) \\
z_P(s)
\end{bmatrix},
\qquad s \in [0,1].
\]

The perception system provides a local surface normal computed from a point cloud obtained by onboard single beam sonar and camera.
\[
n_f \in \mathbb{R}^3,
\]
pointing outward from the net.

From the surface normal, the local net-aligned frame is constructed:
\[
t_f = \frac{n_f \times z_n}{\|n_f \times z_n\|}, 
\qquad
b_f = n_f \times t_f, 
\qquad
z_n = [0,0,1]^T.
\]

Here $t_f$ is the horizontal tangent direction along the net, $b_f$ is the vertical direction
within the net plane, and $n_f$ is the normal.

\subsection{Path Progression Using Waypoints}

The path is provided as a sequence of waypoints
\[
W_0, W_1, \dots, W_N,
\]
which are used only to update the path parameter and determine progression along the
path. 

A waypoint $W_i$ is considered passed when
\[
s_i = (p - W_i) \cdot t_{\text{path}} > 0,
\]
where $t_{\text{path}}$ is the nominal path direction. This allows robust progression even with
uncertain global position estimates.

\subsection{Path Tangent Rotated Into the Net Plane}

The path tangent in the 2D path frame is
\[
\tau_P(s) = \frac{d p_P}{d s},
\qquad
\hat{\tau}_P(s) = 
\frac{\tau_P(s)}{\|\tau_P(s)\|} =
\begin{bmatrix}
\tau_y(s) \\
\tau_z(s)
\end{bmatrix}.
\]

To obtain the corresponding tangent direction on the net surface in 3D, the path tangent is
rotated into the net plane using the basis vectors $(t_f, b_f)$:

\[
\tau_n(s) = \tau_y(s)\, t_f + \tau_z(s)\, b_f.
\]

This vector lies in the net plane and specifies the correct direction of motion for the vehicle.

\subsection{Tangential Motion With Desired Speed}

The vehicle should move along the path with the user-defined desired speed $U_{\text{des}}$.
The desired tangential velocity in the NED frame becomes

\[
v_{\text{path},d}^n = U_{\text{des}}\, \tau_n(s).
\]

This provides the correct combination of horizontal and vertical motion along the net.

\subsection{Orientation Control Using the Surface Normal}

The vehicle’s camera must always face the net. The desired yaw and pitch follow directly
from the surface normal:

\[
\psi_d = \text{atan2}(n_{f,y},\, n_{f,x}),
\qquad
\theta_d = -\arcsin(n_{f,z}).
\]

\subsection{Distance Control Through the Surface Normal}

Tangential motion does not guarantee that the correct distance to the net is maintained.
Therefore, the original surge controller is retained:

\[
e_x = x - x_d,
\qquad
u_d = -K_{p1} e_x - K_{d1} u.
\]

Since the surface normal $n_f$ points outward from the net, the resulting normal-direction
velocity is
\[
v_{n,d}^n = u_d\, n_f.
\]

This pushes the vehicle slightly toward or away from the net depending on the distance error.

\subsection{Combined Desired Velocity}

The complete desired velocity in the NED frame is

\[
v_d^n
= v_{\text{path},d}^n + v_{n,d}^n
= U_{\text{des}}\, \tau_n(s) + u_d\, n_f.
\]

This expression combines path-following motion in the tangent plane with distance control
along the normal. The resulting velocity vector is transformed into the body frame and
tracked by the 6DOF controller.

\subsection{Summary}

The guidance system follows a 2D inspection path defined in the path frame using a fixed
user-chosen speed $U_{\text{des}}$. The path tangent is rotated into the net plane before use,
and the surface normal determines both the orientation and the distance regulation. The
method does not rely on accurate global position tracking and therefore remains robust in
the noisy and uncertain environment near the net surface.
