\chapter{Path Planning}

This chapter describes how the inspection path for the AUV is generated.  
The goal is to cover the entire fish net using a simple and repeatable motion pattern that guarantees full visual coverage.  The AUV will follow the path in 3D. Due to fish pen nets being a flexible structure. They are affected by currents and waves, so their exact position can not be exactly predicted. What we do know is the lenght and witch of the net. I will utilize this to make a 2D path which guarantees full coverage of the net pen. The AUV will follow in 3D. The method described below is designed to be augumented for this purpose.

\section{Reference Frame}

The path is defined in a local \textit{path frame} $\{P\}$, where:
\begin{itemize}
    \item The $z$-axis points down
    \item The $y$-axis points to the right
    
\end{itemize}

For simplicity, the path frame is currently assumed to be the same as the \textit{net frame} $\{N\}$ that follows the fish net surface, which is assumed to correspond to  \textit{North-East-Down frame} $\{NED\}$.
This means that the generated path can be used directly without additional coordinate transformations. In the future one needs to rotate the path into the netframe to follow it.

\section{Motivation}

A common and effective coverage strategy for inspection is the \textit{lawnmower} pattern.  
In this motion, the ROV moves vertically up and down while slowly progressing horizontally between each sweep.  
This pattern ensures that the entire surface is covered if the vertical and horizontal step sizes are smaller than the camera’s field of view.

\section{Analytical Path Generation}

The path is inspired by a unipolar square pulse function, which alternates between two levels.  
A MATLAB implementation of such a signal is:

\begin{lstlisting}[language=Matlab, caption={Unipolar periodic square pulse.}]
A = 10;        % Maximum amplitude
N = 10;        % Number of waves
s = linspace(0, 1, 100);
x = 100 * s;
z = (A/2) * (1 + sign(sin(2*pi*N*s)));
plot(x, z, 'LineWidth', 2);
xlabel('x [m]'); ylabel('z [m]');
title('Unipolar Periodic Square Pulse');
grid on;
\end{lstlisting}

Here, the function switches between 0 and $A$, creating a simple on–off pattern.  
The same concept is used to move the ROV up and down vertically while it gradually moves tangentially along the circumference of the net.

\section{Circular Lawn­mower Pattern in the $y$--$z$ Plane}

Let:
\begin{itemize}
    \item $R$: radius of the net,
    \item $H$: total vertical height of the inspection area,
    \item $N$: number of vertical up–down sweeps.
\end{itemize}

The path length along the net corresponds to one full circle, so the total horizontal distance in the path frame is
\begin{equation}
    W = 2\pi R.
\end{equation}

The analytical path can then be expressed as:
\begin{equation}
    z(s) = \frac{H}{2} \big( 1 + \text{sign}(\sin(2\pi N s)) \big),
\end{equation}
\begin{equation}
    y(s) = W \, s,
\end{equation}
where $s \in [0,1]$ is the path parameter.  
This gives a trajectory where the ROV moves up and down in $z$ while making a full circular sweep along the net’s circumference.

\section{Coverage and Scaling}
Because the path is described analytically, it can easily be adapted to different net sizes.  
By setting the path length equal to the circumference ($2\pi R$) and the height to the total net height $h$, the entire net can be covered.  

To guarantee full coverage, the step sizes in both the horizontal and vertical directions should be smaller than the camera’s field of view. 
