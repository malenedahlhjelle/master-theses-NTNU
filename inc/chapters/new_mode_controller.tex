\chapter{Mode Based Cylindrical Sweep Inspection Controller}

This chapter presents the augmentation from planar nets aligned with NED, to inspection of a deformed cylindrical fish-net. The vehicle follows a lawnmower pattern defined on the net surface while maintaining a constant clearance from the net and aligning its attitude with the net normal. inspired by \cite{Amundsen2021AutonomousROV}\cite{Amundsen2024AquacultureRobotics} \cite{Cardaillac2022IAS}

\section{State Inputs and Geometry Quantities}

At each control step, the controller receives:
\begin{itemize}
    \item Vehicle pose:
    \[
        \eta = 
        \begin{bmatrix}
            x & y & z & \phi & \theta & \psi
        \end{bmatrix}^\top ,
    \]
    \item Body–fixed velocities:
    \[
        \nu =
        \begin{bmatrix}
            u & v & w & p & q & r
        \end{bmatrix}^\top ,
    \]
    \item Cylinder geometry information at the vehicle position:
    \begin{itemize}
        \item $\theta_{\mathrm{rel}}$: angle around the cylinder, where it is assumed the estimation is able to estimate net deformation and movement.
        \item $n_{\mathrm{net}} \in \mathbb{R}^3$: \emph{inward} surface normal of the net.
        \item $s_{\mathrm{clear}}$: clearance to the net along the inward normal:
    \end{itemize}
\end{itemize}

The pattern–following logic maintains:
\begin{itemize}
    \item the current stripe angle $\theta_k$,  
    \item the next stripe angle $\theta_{k+1} = \theta_k + \Delta\theta$,
\end{itemize}
with $\Delta\theta$ defined by the field of view of the camera.

\section{Control Output}

The controller computes the generalized force vector
\[
    \tau = 
    \begin{bmatrix}
        \tau_1 & \tau_2 & \tau_3 & \tau_4 & \tau_5 & \tau_6
    \end{bmatrix}^\top ,
\]
corresponding to body–fixed surge, sway, heave, roll, pitch, and yaw.
All controllers are a form of P, and PID controllers.

\section{Surge Control: Controlling distance to the Net}

The vehicle must remain keep a set distance to the net to allow the camera to see the net, and avoid collsions with the net, this distance is $x_d$. The inward clearance error is
\[
    e_s = \text{distNet} - x_d .
\]

A PID regulator is applied in the surge direction:
\begin{align}
    \tau_1
    &= -K_{p1} e_s - K_{d1} u - K_{i1} \int e_s \, dt .
\end{align}

Since the vehicle's body $x$–axis, this control law ensures that
\[
    \text{distNet} \to x_d \qquad .
\]

\section{Planar Sway and Heave Control}

The planar motion controllers regulate the vehicle either:
\begin{enumerate}
    \item vertically along a stripe ($\theta \approx \theta_k$), or
    \item horizontally along the top or bottom edge 
          ($\theta \to \theta_{k+1}$).
\end{enumerate}

\subsection{Stripe Tracking (Vertical Legs)}

When moving vertically, angle around the net is performed using:
\[
    e_\theta = \operatorname{ssa}\!\left(\theta_{\mathrm{rel}} - \theta_k\right),
\]
with sway control
\[
    \tau_2 = 
        -K_{p\theta} e_\theta 
        - K_{d\theta} v
        - K_{i\theta} \int e_\theta \, dt .
\]

Vertical speed is controlled by a PD law with a depth–dependent reference
\[
    w_{\mathrm{ref}} = 
    \begin{cases}
        \;\;\; w_d \, \alpha(z), & \text{descending},\\[2mm]
        - w_d \, \alpha(z), & \text{ascending},
    \end{cases}
\]
where $\alpha(z) \in [0,1]$ is a slowdown factor near the top/bottom.

Heave control:
\[
    \tau_3 = -K_{p3}(w - w_{\mathrm{ref}}).
\]

\subsection{Horizontal Motion (Bottom/Top Legs)}

When sweeping horizontally between adjacent stripes,
the angle error to the next stripe is
\[
    e_{\theta,\mathrm{next}} =
    \operatorname{wrap}\!\left(\theta_{\mathrm{rel}} - \theta_{k+1}\right).
\]

Sway control drives the vehicle toward $\theta_{k+1}$:
\[
    \tau_2 = - K_{d\theta} (v_d -v).
\]

Depth regulation keeps the vehicle at the top or bottom boundary:
\[
    \tau_3 = 
        -K_{p3z}(z - z_d)
        -K_{d3} w
        -K_{i3} \int (z - z_d)\, dt .
\]

When the condition
\[
    |e_{\theta,\mathrm{next}}| < \varepsilon_\theta
\]
is met, the controller advances the pattern:
\[
    \theta_k \leftarrow \theta_{k+1}, \qquad
    \theta_{k+1} \leftarrow \theta_k + \Delta\theta .
\]

\section{Attitude Control and Net Alignment}\label{sec:attitude}

The inward surface normal is denoted $n_{\mathrm{net}} = [n_x, n_y, n_z]^\top$.
We align the body $x$–axis with this normal using pitch and yaw commands.

A 3–2–1 (yaw–pitch–roll) orientation gives the forward axis:
\[
    a_x =
    \begin{bmatrix}
        \cos\theta \cos\psi \\[2mm]
        \cos\theta \sin\psi \\[2mm]
        -\sin\theta
    \end{bmatrix}.
\]

We desire:
\[
    a_x^{\,d} = n_{\mathrm{net}} .
\]

From the forward–axis equations, the desired pitch and yaw satisfy:
\begin{align}
\theta_d &= \operatorname{atan2}\!\left(-n_z,\, \sqrt{n_x^2 + n_y^2}\right), &
\psi_d &= \operatorname{atan2}(n_y, n_x).
\end{align}

Yaw and pitch errors:
\[
    e_\theta^{\mathrm{att}} = \theta - \theta_d, \qquad
    e_\psi^{\mathrm{att}}   = \psi - \psi_d.
\]

Pitch control:
\[
    \tau_5 =
        -K_{p5} e_\theta^{\mathrm{att}}
        -K_{d5} q
        -K_{i5} \int e_\theta^{\mathrm{att}} dt .
\]

Yaw control:
\[
    \tau_6 =
        -K_{p6} e_\psi^{\mathrm{att}}
        -K_{d6} r
        -K_{i6} \int e_\psi^{\mathrm{att}} dt .
\]

Roll stabilization maintains $\phi \approx 0$:
\[
    \tau_4 = -K_{p4} \phi - K_{d4} p.
\]

\section{Summary of Full Control Law}

The total generalized force vector is
\[
    \tau =
    \begin{bmatrix}
        \tau_1 \\ \tau_2 \\ \tau_3 \\ \tau_4 \\ \tau_5 \\ \tau_6
    \end{bmatrix}.
\]

Each element comes from physically–motivated control objectives:
\begin{itemize}
    \item $\tau_1$: regulate inward clearance from the net,
    \item $\tau_2$: keep constant angle in net, and control speed during horizontal movement  to next stripe.
    \item $\tau_3$: keep constant heave during horizontal movement, and control speed during vertical movement.
    \item $\tau_4$: roll stabilization.
    \item $\tau_5$: pitch alignment with net normal.
    \item $\tau_6$: yaw alignment with net normal.
\end{itemize}

This produces a control law able to follow the net and its deformations.
