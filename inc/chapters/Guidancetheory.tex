\section{Line-of-Sight Guidance for Path Following}

Path following for underactuated marine vehicles such as AUVs is commonly achieved with line-of-sight (LOS) guidance laws. These algorithms generate a desired yaw angle $\psi_d$ (and optionally a desired depth $z_d$) such that the vehicle converges to and follows a straight-line path between waypoints. A large body of work has analyzed LOS-based methods, from proportional LOS \cite{fossen2003los}, integral LOS (ILOS) \cite{borhaug2008integral}, adaptive ILOS \cite{fossen2015adaptive}, and more recently, adaptive LOS (ALOS) \cite{fossen2023alos}.

\subsection{LOS Guidance Principle}
For a straight-line segment defined by two waypoints
\[
P_i^n = (x_i^n, y_i^n, z_i^n), \quad P_{i+1}^n = (x_{i+1}^n, y_{i+1}^n, z_{i+1}^n),
\]
the horizontal path azimuth is
\[
\pi_h = \atan2(y_{i+1}^n - y_i^n,\; x_{i+1}^n - x_i^n).
\]
Transforming the vehicle position $(x_n, y_n)$ into the path-tangential frame gives the cross-track error $y_e^p$:
\[
\begin{bmatrix} x_e^p \\ y_e^p \end{bmatrix}
= R_{z,\pi_h}^\top \left( \begin{bmatrix} x_n \\ y_n \end{bmatrix} - \begin{bmatrix} x_i^n \\ y_i^n \end{bmatrix} \right),
\]
where $R_{z,\pi_h}$ is a planar rotation.

\subsection{Adaptive LOS Guidance Law}
The Adaptive Line-of-Sight (ALOS) law \cite{fossen2023alos} improves robustness against ocean currents by introducing an adaptive estimate of the crab angle $\hat{\beta}_c$. The desired yaw angle is computed as
\begin{equation}
\psi_d = \pi_h - \hat{\beta}_c - \tan^{-1}\!\left(\frac{y_e^p}{\Delta_h}\right),
\label{eq:alos}
\end{equation}
with the adaptation law
\begin{equation}
\dot{\hat{\beta}}_c = \gamma_h \frac{\Delta_h}{\Delta_h^2 + (y_e^p)^2} \, y_e^p,
\label{eq:alos_update}
\end{equation}
where $\Delta_h > 0$ is the lookahead distance and $\gamma_h > 0$ is an adaptation gain.

This law guarantees uniform semiglobal exponential stability (USGES) for straight-line path following at constant speed \cite{fossen2023alos}.

\subsection{Vertical Guidance}
In this thesis, vertical control is simplified by assigning a desired depth $z_d$ directly:
\begin{equation}
z_d = z_i^n + \frac{s}{L}(z_{i+1}^n - z_i^n),
\end{equation}
where $s$ is the along-track distance in the path frame and $L$ is the horizontal path length. This avoids introducing an additional vertical LOS angle and crab compensation in heave, while still ensuring smooth depth transitions.

\subsection{Reasoning for Method Selection}
The ALOS method was chosen because:
\begin{itemize}
    \item It explicitly adapts to unknown and slowly varying sideslip/crab angles caused by ocean currents.
    \item It avoids saturation effects in ILOS, providing better transient performance when disturbances vary.
    \item It has been validated both theoretically (USGES guarantees) and experimentally on AUVs such as the Remus 100 \cite{fossen2023alos}.
    \item For the application of aquaculture net inspection, robustness against variable currents is critical, while vertical motion can be handled by a separate depth controller.
\end{itemize}

Thus, the chosen guidance structure combines ALOS in the horizontal plane with direct depth assignment in the vertical plane, ensuring robustness and practical implementability in 3D missions.

\bibliographystyle{ieeetr}
\bibliography{references}
