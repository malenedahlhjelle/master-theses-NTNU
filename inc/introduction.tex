\chapter{Introduction}
\label{chap:introduction}
Aquaculture is one of the fastest-growing food production industries worldwide, and fish farming nets represent a critical infrastructure for safe and sustainable operations. Structural failures such as tears or holes in the nets can cause major economic and environmental consequences, making reliable inspection and maintenance essential. While traditional inspection methods rely on divers or manually operated ROVs, these approaches are resource-intensive and expose operations to both safety and environmental risks. Autonomous underwater systems offer a promising alternative by reducing human involvement, ensuring systematic coverage, and providing accurate localization of damages.

Mohn Technology is currently developing an autonomous underwater vehicle (AUV) designed to perform full inspections of aquaculture nets. The vehicle is equipped with several onboard sensors, including stereo cameras, an IMU with compass, a pressure sensor for depth, and a single-beam sonar for net distance measurement. While low-level motion control is already available through the AUV’s autopilot, robust high-level guidance methods are required to enable safe and complete inspection missions. In particular, solutions must account for the cylindrical geometry of the fish pen, sensor-based navigation, and operational challenges such as temporary loss of position estimates.

This thesis focuses on the guidance system within the overall autonomy architecture, illustrated in \ref{fig:blokkdiagram}. The work involves developing and simulating path planning algorithms, risk-aware coverage strategies, and guidance laws that generate feasible state references for the AUV’s existing control system.

\begin{figure}
    \centering
    \includegraphics[width=\textwidth]{figures/blokkdiagram.drawio-2.png}
    \label{fig:blokkdiagram}
    \caption{High-level block diagram}
\end{figure}{}
Key objectives include:

Ensuring full coverage of the net wall through efficient and feasible path planning.

Handling temporary positioning loss by enabling recovery strategies that preserve inspection progress.

Integrating inspection data collected during recovery phases to avoid coverage gaps and redundant operations.

Designing a line-of-sight (LOS) based guidance law that outputs course and depth commands suitable for direct autopilot input.

All work in this thesis is conducted within a simulation framework, representing the AUV through a closed-loop kinematic model and sensor setup. Validation is achieved through simulation studies, and the results are visualized in Mohn Technology’s Unity-based simulator. By focusing on the guidance layer, this work contributes to the development of autonomous methods that improve the robustness, reliability, and efficiency of net inspection in aquaculture applications.
% \q{What is the purpose of this document?}
% The purpose of the current document is to provide an example of using the thesis template and a description on how to use \LaTeX. There are instructions that relate specifically to the template, and some which are generally useful for \LaTeX. 

% \q{Where is the structure defined } 
% The detailed typographical rules have been implemented in the
% \texttt{ntnuthesis} \LaTeX\ document class, in the file 

% This was updated by Simon McCallum.
% The package has changed names and version number it is now called
% \texttt{ntnuthesis}
% v\ntnuthesisversion\ as of \ntnuthesisdate.

% \q{what is a thesis about}
% For many of you, this Master's thesis will be the most advanced academic document you ever write.  It needs to demonstrate both academic ability and clear thinking. You Master's should show that you are ready to lead other people, reflect more deeply, and have a professional attitude to your work and environment. 

% \q{who cares}
% When writing the thesis it is important to know who you are writing for. The target audience for this document is in layers:
% \begin{enumerate}
%     \item The marking committee
%     \item Your supervisor
%     \item Other students at the same level 
%     \item Professionals \& Academics
%     \item The general public.
% \end{enumerate}
